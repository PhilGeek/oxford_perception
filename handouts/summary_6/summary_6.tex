%!TEX TS-program = xelatex 
%!TEX TS-options = -output-driver="xdvipdfmx -q -E"
%!TEX encoding = UTF-8 Unicode
%
%  summary_6
%
%  Created by Mark Eli Kalderon on 2009-03-23.
%

\documentclass[11pt]{article} 

% Definitions
\newcommand\myauthor{Mark Eli Kalderon} 
\newcommand\mytitle{Oxford Philosophy of Perception:}
\newcommand\mysubtitle{G.A. Paul and the Problem of Sense Datum}

% Packages
\usepackage{url}
\usepackage{txfonts}
\usepackage{mdwlist}
\usepackage{color}
\definecolor{myblue}{rgb}{0.8,0.8,1}

% Define discussion environment
\makeatletter\newenvironment{discussion}{%
   \noindent\begin{lrbox}{\@tempboxa}\begin{minipage}{\columnwidth}\setlength{\parindent}{1em}}{\end{minipage}\end{lrbox}%
   \colorbox{myblue}{\usebox{\@tempboxa}}
}\makeatother

% XeTeX
\usepackage[cm-default]{fontspec}
\usepackage{xltxtra,xunicode}
\defaultfontfeatures{Scale=MatchLowercase,Mapping=tex-text}
\setmainfont{Hoefler Text}
\setsansfont{Gill Sans}
\setmonofont{Inconsolata}

% Title Information
\title{\mytitle\\
\mysubtitle}
\author{\myauthor} 
\date{} % Leave blank for no date, comment out for most recent date

% PDF Stuff
\usepackage[plainpages=false, pdfpagelabels, bookmarksnumbered, backref, pdftitle={\mytitle}, pagebackref, pdfauthor={\myauthor}, xetex, colorlinks=true, linkcolor=gray, urlcolor=gray]{hyperref}

%%% BEGIN DOCUMENT
\begin{document}

% Title Page
\maketitle

% Layout Settings
\setlength{\parindent}{1em}

% Main Content

\section{Cambridge Realism} % (fold)
\label{sec:Cambridge_realism}
Paul's primary target is Moore's conception of sense perception and Cambridge realism more generally.

Cambridge realists shared a conception of sense perception with their Oxford counterparts. Roughly speaking, sense perception is conceived as a fundamental mode of awareness of the mind-independent environment, a mode of awareness that enables a subject with the appropriate recognitional capacities to have propositional knowledge of that environment. Two theses that frame this realist conception of perception are presently important, specifically:
\begin{enumerate}
    \item The objects of knowledge are independent of the act of knowing.
    \item Sense perception is a form of knowing. 
\suspend{enumerate}

The first thesis is the fundamental realist (or anti-idealist) commitment common to both Moore and Cook Wilson---that the objects of knowledge could not constitutively depend on acts of knowing as the idealist maintained.

The second thesis links the fundamental realist commitment to the objects of perception. If perception is a form of knowing and the objects of knowledge are independent of the act of knowing, then the objects of perception are independent of the act of perceiving. That sense perception is a form of knowing was held not only by the Oxford realists such as Cook Wilson and the early Prichard but also by Cambridge realists such as Moore and Russell, at least in certain incarnations. 

In what sense is perception a form of knowing? Here are two interpretations. They are conceptually and extensionally nonequivalent; nevertheless, they are consistent and potentially complementary:
\begin{enumerate}
    \item Knowledge is the most general category of certain cognitive attitudes (cognitive in the sense that contrasts with conative and not in the sense that these attitudes are analyzable or otherwise reducible to beliefs.) In contrast with belief, this kind of cognitive attitude takes either propositional or non-propositional objects. In further contrast with belief, the relevant propositional attitudes are all \emph{factive}. The relevant nonpropositional attitudes are all object-involving in the sense that one can only have these attitudes to an object if that object exists. (Compare Russell's distinction between \emph{knowledge by acquaintance} and \emph{knowledge by description}. Russell once held that sense perception was a paradigm example of knowledge by acquaintance.)
    \item Sense perception is a fundamental, sensory mode of awareness that relates a subject to an object in the mind-independent environment. In being so related, the subject is \emph{knowledgeable} of that environment. In being so aware of an object, the subject is in a position to have propositional knowledge about that object depending on the subject's possession and exercise of recognitional capacities appropriate to the circumstances. Perception is a form of knowing in the sense that it makes the subject in this sense knowledgeable of its object.
\end{enumerate}
Either interpretation suffices to link the fundamental realist (or anti-idealist) thesis to the objects of perception with the consequence that the objects of perception are independent of our awareness of them in sense perception. 

Cambridge realism departs from Oxford realism in its adherence to a further thesis. Let sense experience be the genus of which sense perception is a species. Cambridge realists, in addition, held:
\resume{enumerate}
    \item There is something of which a subject is aware in undergoing sense experience.
\suspend{enumerate}
According to the theories of Russell, Moore, and Price, sense data are whatever we are aware of in sense experience. So understood, sense data just are whatever entities that play this epistemic role. This characterization of sense data is \emph{neutral} in the sense that it assumes nothing about the substantive nature of objects that play this epistemic role. Further argument is required to establish substantive claims about the nature of sense data. So for example, it was typically felt that the generalizing step in the argument from illusion---that a subject is aware of the same kind of thing in perceptual and nonperceptual experience, say---requires further, if swift, argument.

Paul's primary target is a consequence of the realist conception of knowledge, the thesis that sensing is a form of knowing, and the ``neutral'' characterization of sense data:
\resume{enumerate}
    \item Sense data are objects independent of our awareness of them in sense experience whose substantive nature is open to investigation.
\end{enumerate}

% section Cambridge_realism (end)

\section{Is There a Problem about Sense Data?} % (fold)
\label{sec:is_there_a_problem_about_sense_data_}
In ``Is There a Problem about Sense Data'' Paul makes negative and positive claims:
\begin{itemize}
    \item Paul denies that sense data are objects independent of our awareness of them in sense experience whose substantive nature is open to investigation.
    \item Paul claims instead that sense data are shadows cast by experiences that can elicited by suitably affecting the mind.
    \item Open questions about the nature of sense data are resolved not by investigation but by linguistic decision.
\end{itemize}

If sense data are independent of our awareness of them in undergoing sense experience, then open questions about their substantive nature should be settled by looking to the sense data themselves. But questions about their identity and persistence conditions are not settled in this way. In so far as we \emph{can} frame answers to questions about sense data, these answers are in terms of what appears to a subject in undergoing a sense experience in a given circumstance. Disagreements arising from open questions about the nature of sense data, such as whether sense data can persist unsensed, are really practical disagreements about how to recast our ordinary appearance talk in terms of talk about sense data.

Here we encounter an alien influence on Oxford reflection on perception---in the present instance, a Viennese influence. Indeed, ``Is There a Problem about Sense Data'' was composed prior to Paul's coming to Oxford while he was still in Cambridge where he studied and worked with Wittgenstein. 
Paul and Ayer can be read as taking over from Carnap and the other positivists the general idea that there is no substantive metaphysics and that metaphysical disagreements are better understood as practical disagreements about what language or conceptual scheme to adopt. Paul and Ayer apply this general idea to sense data and suggest that talk of sense data is just an alternative way of talking about facts that all of us can agree about, namely, facts about appearances. The application of the general idea to sense data can be found in Wittgenstein, and Wittgenstein's middle period discussion of sense data is the most likely proximate influence on Paul:\begin{quote}
    Philosophers say it as a philosophical opinion or conviction that there are sense data. But to say that I believe that there are sense data comes to saying that I believe that an object may appear to be before our eyes even when it isn't. Now when one uses the word ``sense datum", one should be clear about the peculiarity of its grammar. For the idea in introducing this expression was to model expressions referring to `appearance' after expressions referring to `reality'. It was said, e.g., that if two things seem to be equal, there must be two somethings which are equal. Which of course means nothing else but that we have decided to use such an expression as ``the appearances of these two things are equal" synonymously with ``these two things seem to be equal". Queerly enough, the introduction of this new phraseology has deluded people into thinking that they had discovered new entities, new elements of the structure of the world, as though to say ``I believe that there are sense data" were similar to saying ``I believe that matter consists of electrons". (\emph{The Blue Book}, 70)
\end{quote}
% When we talk of the equality of appearances or sense data, we introduce a new usage of the word ``equal". It is possible that the lengths A and B should appear to us to be equal, that B and C should appear to be equal, but that A and C do not appear to be equal. And in the new notation we shall have to say that though the appearance (sense datum) of A is equal to that of B and the appearance of B equal to that of C, the appearance of A is not equal to the appearance of C; which is all right if you don't mind using ``equal" intransitively.

% While doubts about the substantive nature of sense data had been entertained before, what is distinctive about Paul (and Ayer after him) is that he regards open questions about the nature of sense data not as questions of substantive fact but as matters for linguistic decision. Sense datum talk is a regimentation of our talk of what appears to us in undergoing sense experience and inherits whatever content it has from the experiential reports it regiments. 


``Is There a Problem about Sense Data'' divides into two parts. The first part concerns the existence of sense data. The second part concerns the identity and persistence conditions of sense data.
% section is_there_a_problem_about_sense_data_ (end)

\section{The Existence of Sense Data} % (fold)
\label{sec:the_existence_of_sense_data}

Paul states his conclusion about the existence of sense data as follows:
\begin{quote}
    My intention has not been to deny that there are sense-data, if by that is meant that (1) we can understand, to some extent at least, how people wish to use the word `sense-datum' who have introduced it into philosophy, and that (2) sometimes statements of a certain form containing the word `sense-datum' are true, e.g., `I am seeing* an elliptical sense-datum ``of'' a round penny.' Nor do I wish to deny that the introduction of this terminology may be useful in helping to solve some philosophical problems about perception; but I do wish to deny that there is any sense in which this terminology is nearer to reality than any other which may be used to express the same facts; in particular I wish to deny that in order to give a complete and accurate account of any perceptual situation it is necessary to use a noun in the way in which `sense-datum' is used \ldots\ (``Is There a Problem about Sense-Data?'')
\end{quote}
According to Paul, then: 
\begin{enumerate}
    \item Claims about sense data are intelligible.
    \item Claims about sense data, intelligibly interpreted, can be true.
    \item Accepting true claims about sense data does not commit one to the existence of mind-independent sense data.
\end{enumerate}
Let's consider these claims in turn.

Paul, like Wittgenstein, claims that talk of sense data is a theoretically motivated regimentation of ordinary appearance talk. In ordinary appearance talk we report how things appear in undergoing various sense experiences in various circumstances of perception. Paul specifies the particular sense of ``appearance'' and related idiom that talk of sense data is introduced to regiment:
\begin{quote}
    We so use language that whenever it is true that I am seeing the round top surface of a penny, and know that it is round, it is true to say that the penny \emph{looks} (e.g.) elliptical to me, in a sense in which this does not entail that I am in any way deceived about the real shape of the penny. (I shall indicate this sense by means of a suffix: `looks*'.)
\end{quote}
Starred look statements are not Jackson's \emph{epistemic look statements}. An epistemic look statement introduces a thought as a topic of conversation and asserts that there is visual evidence for its truth. It is a necessary condition on the assertability of an epistemic look statement that the conversationally salient thought be epistemically possible in the circumstances. But in Paul's example, the penny is seen to be round. And if it is seen to be round, it is evidently not elliptical. In the given circumstances of perception, it is not epistemically possible that the penny be elliptical. So the intended sense of ``looks* elliptical'' is not the epistemic sense. Starred look statements must at least mark the appearance/reality contrast in the following sense: Something can look* \( F \) without itself being \( F \), indeed while seen to be not-\( F \). 

\begin{discussion}
	Ryle and Chisholm will propose that the comparative use of look statements can mark such a contrast. On the comparative use, the penny looks elliptical roughly in the sense that it looks the way an elliptical thing would look in some other contextually salient circumstance. Moreover, it is a look that the penny can manifest in sense experience even when that penny is seen to be round. Paul does not commit himself to the comparative interpretation, however.
\end{discussion}

Sense data talk is explained in terms of starred appearance attributions in accordance with the rule that the adjectival complement of a starred appearance attribution is predicated of the sense datum. So if something looks \( F \) in a subject's experience of it, then the subject has an \( F \) sense data. In this way, ordinary appearance attributions, in an intended sense, control talk of sense data, and talk of sense data derives whatever content that it has from the intended sense of ordinary appearance attributions.

\begin{discussion}
    While the transformation rule that Paul proposes is initially plausible, can all \emph{intelligible} talk of sense data be understood, without semantic residue, in terms of starred appearance attributions? Importantly, Paul claims that not \emph{all} sense data talk is intelligible---but only because its use is not settled by the intended sense of ordinary appearance attributions. For that to be plausible we must be antecedently convinced that the intelligible use of sense datum talk can be \emph{completely} explained in terms of starred appearance attributions.
\end{discussion}

So far I have explained the sense in which Paul believes that claims about sense data can be intelligibly interpreted. Such claims are intelligibly interpreted as restatements of our ordinary appearance attributions in their non-epistemic sense. Since these ordinary appearance attributions can be true, so can their restatements in terms of sense-data. So Paul's second claim is a consequence of his first claim and the commitments of common sense that we bring with us to philosophy.

Given how talk of sense data must be intelligibly interpreted, Paul is in a position to provisionally argue for his third claim---that accepting true claims about sense data does not commit one to the existence of mind\-in\-de\-pen\-dent sense data. Specifically, if claims about sense data are only intelligibly interpreted as restatement of ordinary appearance attributions in their non-epistemic sense, then there are no facts of the matter about sense data over and above how things appear to a subject in undergoing a sense experience. But facts about how things appear in sense experience are not facts about a domain of mind-\-independent entities. 

Paul's three conclusion are largely driven by an epistemological contrast. If sense data are objects of investigation independent of our awareness of them in sense experience, question about their existence should be addressed to the sense data themselves. If, on the other hand, sense data are shadows cast by our experiences as outlined by a language that we contingently choose to speak, then questions about their existence are not settled by addressing them as independent objects of investigation but by the appearances that cast them.

To illustrate the contrast Paul asks us to compare questions about foveas with questions about sense data. ``Fovea'' is a technical term that could have been introduced by means of a description--- a fovea is the small depression on the retina corresponding to the pupil. So the difficulty is not due to the fact that ``sense data'' is a technical term. Foveas are independent objects of investigation. We can determine whether anything satisfies the description by empirical investigation---we look and see whether this depression exists by dissecting an eye. But we know what it would be like to discover such a depression when we look. And this presupposes that we know what it would be like to discover that there is no such depression when we look. Sense data, in contrast, we see whenever we look. We don't know what it would be like to be aware of sense data in sense experience since we don't know what it would be like not to be aware of sense data in sense experience. Claims about sense data seem to simply to be restatements of how things appear in undergoing sense experience. So claims about sense data are settled, where they are, by how things appear in undergoing various sense experiences in various circumstances of perception.

\begin{discussion}
    Doubts may be raised about Paul's case. Too an extent, these doubts can be addressed given the argument of the second part of Paul's essay.
    
    First, following Alston and Wright (and Frege before them), one might question the reductive character of Paul's interpretation of sense-datum talk. Perhaps talk of sense data, in the familiar Fregean metaphor, \emph{reconceptualizes} ordinary appearance attributions in their non-epistemic sense, and in so doing makes explicit what was merely an implicit commitment to the existence of sense data in our ordinary appearance attributions.
\end{discussion}

\begin{discussion}
    It can seem that Paul's provisional argument against sense data conceived as mind-independent objects of investigation presupposes phenomenalism about appearances. Attributions of appearances must be understood in terms of actual and hypothetical experiences---in terms of how something does or would look or appear to a subject that undergoes the relevant sense experience in the given circumstance of perception. Suppose that phenomenalism about appearances is false. There may be no facts about sense data over and above facts about non-epistemic appearances, but if facts about appearances are not facts about actual and hypothetical sense experience, then there is not yet reason to deny that sense data are intelligibly understood as mind-independent entities.

    Not only does Paul's provisional argument seem to presuppose phenomenalism about appearances, it seems to presuppose, as well, an optional and controversial conception of perceptual experience---at least it was natural for early twentieth century realists to reject it. Specifically, in claiming that sense data are shadows cast by experiences that can be elicited by suitably affecting the mind, perceptual experience is understood as a way of being affected, as a conscious modification of the perceiving subject. A change in objects spatially external to the subject does not in this sense modify the perceiving subject. So perceptual experience, conceived as a conscious modification of the subject, does not constitutively depend on external objects. (This is roughly Ducasse's conception of experience and is required by phenomenalism about appearances.) Oxford and Cambridge realists, in contrast, endorse a relational conception on which perceptual experience constitutively depend on external object that the subject is aware of. The relational conception can be motivated by a digestive metaphor for sensory awareness---it is the means by which we take in the objects of the external environment.
    
    In the second part of the essay, Paul provides an additional and more powerful argument against the mind-independence of sense data. If sense data are what are sensed, and in sensing we know mind-independent entities, the sense data are mind-independent. But if sense data are mind-independent, then there must be substantive facts about the identity and persistence-conditions of sense data. And it is precisely this that Paul doubts. Two observations. First, this argument does not presuppose phenomenalism about appearances, nor does it presuppose that experience is a conscious modification of the perceiving subject. Second, even if sense data were an implicit commitment of ordinary appearance attributions they could not be entities open to investigation independent of our awareness of them.
\end{discussion}

% section the_existence_of_sense_data (end)

\section{The Identity and Individuation of Sense Data} % (fold)
\label{sec:the_identity_and_individuation_of_sense_data}

According to the conception of sense perception common to Oxford and Cambridge realists, sensing is a form of knowing, and the objects of knowledge are independent of the act of knowing. Paul doubts whether this conception of sense perception can be sustained if we further assume that we are aware of sense data in undergoing sense experience. The resulting indirect realism of Moore naturally raises a veil of perception worry. If what I am directly aware of in sense perception is sense data and not material objects, then how on the basis of sense perception can we be in a position to know about material objects? Moore's hypothesis (that he struggled to maintain throughout his career) that sense data are material surfaces, might provide an answer. Since material surfaces are not independent of the material objects whose surfaces they are, it is plausible that sensuous awareness of material surfaces makes you knowledgeable of material objects. However, this response to the veil of perception worry is committed to there being determinate answers to questions that, according to Paul, so far lack sense.

Specifically, in claiming that sense data are material surfaces, Moore is committed to there being determinate answers to whether sense data are public or private, whether sense data can persist unsensed, and whether sense data can be reencountered in distinct acts of awareness. If sense data are material surfaces, then sense data are public, can persist unsensed, and can be reencountered in distinct sensings. There is no one subject that one must be to be aware of a material surface, material surfaces persist whether a subject is sensuously aware of it or not, and material surfaces can be reencountered in distinct sense experiences.

Paul assumes that material objects exist independently of our sensory awareness of them. Indeed, Paul sometimes writes as if part of what it means to be material is to exist independently of the mind. So Paul's main point is about the alleged \emph{mind-independence} of sense data.

Paul begins with some preconditions for an object to be open to investigation independent of our awareness of it. Specifically, if an object exists independently of a subject's awareness of it, then it is at least intelligible:
\begin{enumerate}
    \item the object is public in the sense that there is no one subject that one must be to be aware of it,
    \item that the object can persist when no subject is aware of it, and
    \item that the object can be reencountered in distinct acts of awareness.
\end{enumerate}
If an object exists independently of a subject's awareness of it in perceptual experience, then it is at least intelligible to suppose that another subject, viewing the same scene, could be aware of it in their perceptual experience of that scene. Moreover if an object is independent of a subject's awareness of it, it is at least intelligible to suppose that the object persists when no one is aware of it. Further if the object of awareness is independent of a subject's awareness of it, then it is intelligible that the subject may reencounter that object in distinct episodes of sensory awareness. Part of what it is to think of an object of awareness as being independent of that awareness is just for it to be intelligible that the object is public, can persist independently of our awareness of it, and can be reencountered in distinct acts of awareness. 

It is only required that these things be \emph{intelligible}. For example, a mind-independent object might, but need not, be temporally coincident with a subject's awareness of it. So it is not a requirement for an object to have the requisite mind-independence that it exists at a time when no one is aware of it. But part of what it is to think of that object as independent of our awareness is for it to be intelligible that the object exists when no one is aware of it. It is partly because it is intelligible to suppose that the object exists when no one is aware of it, that we can entertain the modal judgment that the temporal coincidence is contingent. Moreover the intelligibility of the thought that the object exists when no one is aware of would survive the discovery of temporally coincidence. Indeed, the discovery that the object is temporally coincident with a subject's awareness of it presupposes that it is intelligible that the object exists when no one is aware of it. If it weren't intelligible, it couldn't be the object of discovery. 

More generally, we can only intelligibly hold that an object is independent of a subject's awareness of it if it is intelligible not only that the object exist when no one is aware, but that it be public and be reencountered in distinct acts of awareness as well.

Given these preconditions for an object to be mind-independent, Paul argues sense data could not be independent of our awareness of them in sense experience. The argument proceeds by \emph{modus tollens}. If sense data were independent of our awareness of them in sense experience, it would be intelligible that they exist unsensed, are public, and can be reencountered in distinct courses of sense experience. But these things aren't intelligible:  
\begin{quote}
    [S]uppose I look at the round surface of the penny from a certain angle, then shut my eyes or go away for five minutes, then look at it again under similar conditions from the same place, we might describe this correctly by saying that ``I saw two different appearances of it which were exactly the same in shape and color''. We do not describe it by saying that ``I saw twice the same appearance which continued to exist during the period when I wasn't seeing it''. Whether we are to say that in this case I saw numerically the same sense-datum twice over or that I saw sense-data which had the same qualities is a matter of indifference \ldots\
\end{quote}
And so we can't intelligibly regard sense data as possessing a substantive nature open to investigation independent of our awareness of them. Moreover, if we cannot intelligibly regard sense data as objects open to investigation independent of our awareness of them, we cannot identify sense data with material surfaces as Moore recommends. And if sense data are not material surfaces, then we cannot overcome the veil of perception worry in this way.

That it is unintelligible to claim that sense data exist when no is aware of them means, in the present instance, that such claims have no use. This is not, however, to say that a use may not be assigned to such claims. But that is a matter for linguistic decision, not discovery:
\begin{quote}
    The important point is whatever we do is not demanded by the nature of objects which we are calling `sense-data', but that we have a choice of different notations for describing observations, the choice being determined only by the greater convenience of one notation, or our personal inclination, or by tossing a coin.
\end{quote}
(In the \emph{Blue Book}, Wittgenstein sometimes uses ``notation'' as a terminological variant of ``language game''.) So even if we were to extend sense data talk so that it made sense to say that a sense datum exists unsensed, this would just what we would \emph{call} an unsensed sense data, not the discovery of unsensed sense data.

% section the_identity_and_individuation_of_sense_data (end)

\section{Some Consequences} % (fold)
\label{sec:some_consequences}

Suppose Paul's central claim is right---that sense data do not have a substantive nature open to investigation independent of our awareness of them in sense experience. There are at least three potential morals:
\begin{enumerate}
    \item One might deny that there are any substantive facts about the nature of sense data that are open to investigation. (Paul, Ayer)
    \item One might claim that sense data constitutively depend on our awareness of them in sense experience. Sense data would be, in this regard, like Berkelean ideas. So conceived, sense data would lack a substantive nature independent of our awareness of them; though, Ayer, at least, would regard this Berkelean alternative as piece of substantive metaphysics on a par with Moorean sense data. (Though neither deploy the sense data vocabulary, Berkeley, later Prichard)
    \item One might retain the conception of perception common to Oxford and Cambridge realists by abandoning the fundamental claim of the sense-datum theory---that there is an object of which we are aware whenever we undergo sense experience. (Austin, Hinton)
\end{enumerate}

% section some_consequences (end)

\end{document}
